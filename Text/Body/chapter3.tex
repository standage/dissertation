\chapter{SCALABLE GENOME ANNOTATION FOR PROVISIONAL GENOME ASSEMBLIES}

A paper published in BMC Genomics/Genome Biology: url.

\section{Introduction}

Lorem ipsum dolor sit amet, consectetur adipiscing elit. Vivamus ac feugiat mauris. Nunc sed felis a purus finibus cursus in eu ligula. Nam cursus iaculis augue eget rutrum. Curabitur sed lorem posuere, ultricies nisl ac, dictum est. Praesent accumsan urna turpis, nec tristique nulla rutrum sollicitudin. Vivamus eu sapien id risus fringilla faucibus. Vestibulum euismod, nibh nec rutrum interdum, ex urna vehicula lorem, et fermentum tortor nunc at nisi. Curabitur urna metus, suscipit a ipsum ac, consectetur pharetra augue. Sed sit amet turpis vel risus vehicula dapibus. Duis mattis metus tellus, sit amet placerat lacus tincidunt ut. Nullam dictum lacus magna, in porttitor elit malesuada nec. Quisque quis massa luctus dui tincidunt hendrerit vel eget nisl.

Suspendisse et massa dolor. Cras cursus finibus enim in dapibus. Morbi aliquet placerat arcu, sed tristique ante pulvinar nec. Proin non metus non felis imperdiet tristique tristique vel augue. Cras posuere condimentum purus, vitae tempus tellus. Sed nibh velit, scelerisque vitae felis sit amet, dignissim sollicitudin tellus. Nam eget lacus vitae dolor fermentum fermentum id id magna. Donec auctor euismod porta. Cras in ante scelerisque, placerat enim eu, dictum nisi. Integer nunc eros, elementum tempor arcu sed, tristique hendrerit leo.

As can be seen in Table~\ref{nothing} it is truly obvious what I am saying is true.

\begin{table}[h!tb] \centering
\isucaption{This table shows a standard empty table}
\label{nothing}
\vspace{ 2 in}
\end{table}

Class aptent taciti sociosqu ad litora torquent per conubia nostra, per inceptos himenaeos. Cras tincidunt vehicula mi in ultrices. Proin vitae mauris aliquam, rutrum mi non, maximus libero. Cras sit amet metus sit amet nisi posuere eleifend. Nam et sapien odio. In ultrices elit nibh, sit amet commodo purus lobortis vitae. Quisque ac felis interdum, ornare nulla fringilla, posuere augue. Nullam dictum et arcu non ornare. In faucibus hendrerit nibh nec mollis. Nam eu dolor sodales mauris fermentum ornare ac ac ante. Etiam non odio sed odio faucibus luctus sed sed nulla. Aliquam sit amet est bibendum, lacinia velit eget, ornare mi. Nam eros neque, scelerisque quis cursus egestas, placerat eu nisl. Vivamus scelerisque odio at ipsum faucibus faucibus. Mauris consequat eu felis nec vulputate.

Fusce finibus erat nulla, eget vestibulum diam tristique ac. Fusce nisi diam, finibus vitae fermentum nec, placerat sodales tellus. Praesent et accumsan nunc. Pellentesque quam orci, rutrum quis ultricies quis, facilisis a ante. Curabitur felis ex, efficitur ut blandit eu, luctus quis enim. Aliquam ac lacinia massa. Quisque aliquam, quam at aliquam venenatis, magna ante auctor purus, nec pharetra turpis urna et diam. Duis eu lectus eget risus ultrices lacinia. Quisque tincidunt purus ac nunc ornare, at pharetra erat rutrum. Sed massa sem, iaculis at vestibulum eget, accumsan eu nibh.

Pellentesque habitant morbi tristique senectus et netus et malesuada fames ac turpis egestas. Sed lacus augue, euismod sed lacinia at, rutrum eget lectus. Donec egestas massa ac risus finibus mattis id eu velit. Mauris fermentum ligula vel tempor mattis. Etiam vehicula arcu a venenatis elementum. Ut cursus molestie ex eget auctor. Morbi eget risus a purus sodales semper. Quisque efficitur laoreet nunc, interdum volutpat orci. Sed quam ligula, dignissim sed dui vel, finibus ultricies elit. Praesent ipsum lectus, finibus sit amet mattis vitae, tempus non turpis. Nulla facilisi. Curabitur et dignissim nibh.


\section{Hypothesis}

Here one particular hypothesis is explained in depth
and is examined in the light of current literature.

This can also be seen in Figure~\ref{moon} that the
rest is obvious.

\begin{figure}[h!tb] \centering

\vspace{ 2 in}
\isucaption{This table shows a standard empty figure}
\label{moon}
\end{figure}
