\chapter{CONCLUSIONS}

A draft genome assembly and corresponding anntotation are important data resources for any genome project, and provide the coordinate systems and catalogs of functional elements upon which genomics studies rely.
At the same time, it is widely acknowledged that genome assembly and annotation are complex and difficult with current nucleotide sequencing technologies, requiring close collaboration between domain experts and bioinformaticians.
Single-molecule long-read sequencing platforms promise exciting improvements genome and transcriptome assembly, and within a few years could facilitate a much more direct path to production of high-quality reference genome sequences.
As the cost and effort corresponding to genome-scale sequencing continues to drop, the demands imposed on domain experts to interpret genome-scale data will only continue to increase.
The focus of this dissertation has been the development of tools and methodologies that enable reproducible analysis of annotated genomes, as motivated by challenges encountered in assembling, annotating, and analyzing the genome of the paper wasp \textit{Polistes dominula}.

\textit{Polistes} is an important model for the study of social behavior and its evolution.
The genome of \texit{Polistes} is similar to other social insects in many characteristic features, and shows a slight reduction in others, such as intron length and genome size \cite{PdomGenome}.
The nucleotide composition of \textit{Polistes dominula} is the most AT-rich of any social insect studied, raising compelling questions about the molecular evolution of the genome and possible contributions of biased repair/mutation mechanisms or historically elevated levels of DNA methylation.
Most striking in terms of genome content is the conspicious absence of the DNA methyltransferase \textit{Dnmt3}, recently and independently confirmed as missing in another \textit{Polistes} genome \cite{PcanGenome}.
Previously thought to be a hallmark signature of social behavior \cite{Glastad2011}, \textit{Polistes} is the first social insect known to lack a complete complement of DNA methylation enzymes.
We also found diffexp.

The challenges of this particular project are easily generalized, and I therefore sought to create tools that could be applied generally to address common questions about genome composition and organization.
Foremost, researchers need to be able to scan newly produced genomes for genes of interest---for example, established ``social behavior" genes or DNA methylation related enzymes in \textit{Polistes}.
Researchers must also be able to compare different sources of annotation for the genome, and assess these in the context of available expression data and other evidence.
There must be a way for researchers to refine annotations at loci of interest as needed, and to track changes to the annotation over time, providing complete providence for each data point included in an analysis.
And finally, researchers need to be able to select subsets of data with which to reliably describe characteristics of the genome, such as (for example) intron length or codon usage.

The framework provided by \textit{interval loci (iLoci)} and related tools is designed to provide precise solutions to these challenges.
iLoci furnish a ``parts list" of a genome, a representation of the genome that is granular, complete, and that captures the genomic context of each genic or intergenic region.
The ParsEval tool \cite{ParsEval} identifies the extent to which alternative sources of annotation for the genome differ, and in concert with community annotation tools (such as those provided by xGDBvm \cite{xGDBvm}) facilitates targeted evaluation and refinement of genes of interest.
iLocus statistics produced by GenHub \cite{GenHub} enable filtering of iLoci based on their length, nucleotide composition, and gene content.
A prototype of a related tool, GeneAnnoLogy, implements a version control system to track changes in annotations over time, and could be extended with more flexible and comprehensive iLocus filtering mechanisms.

The tools described in this dissertation are designed to calculate and present characteristics of genome composition quickly and easily for a new genome, and to facilitate uniform re-analysis of related data sets for comparison.
Accordingly, care was taken to ensure that the tools support standard data formats, and to reduce the friction involved in installing and executing the software.
These practical concerns, while largely orthogonal to the science, can nevertheless make ``quick and easy" processing and---ultimately---interpretation prohibitively difficult.
Just as improvements in DNA sequencing technology have democratized genome sequencing, making these tools accessible and compatible with well-defined standard data formats puts the power of genome analysis directly in the hands of domain experts.
Rapid assessment of new genome assemblies and annotations in the context of related genomes will help researchers distinguish between genomic features that are broadly conserved and those that appear to be unique, warranting additional attention and scrutiny.

% we also want/need to learn more about organization.







% The focus of this dissertation has been the development of tools and methodologies for reproducible analysis of genome annotations, including and especially those associated with NGS-based genome projects.
% The motivation for much of this work was provided by challenges encountered during the assembly, annotation, and analysis of the genome of the paper wasp \textit{Polistes dominula}.
% In addressing these challenges, we sought to create tools that could be applied more generally to address common questions about genome composition, and organization.
%
% In the \textit{P. dominula} genome project, we developed a variety of genomic data resources \textit{de novo} for investigating genome composition, gene expression, and characteristics of DNA methylation.
% Our study revealed that the paper wasp genome is similar to other social insects across a variety of measures: genome size, number of annotated genes, and exon and intron characteristics, but that its nucleotide composition is the most AT-rich (aggregated over large genomic sequences) of any hymenopteran studied.
% We identified 367 loci in the \textit{P. dominula} genome that are differentially expressed between queen wasps and worker wasps, and though we catalogued thousands of alternative splicing events we found no evidence of differential splicing between adult queens and workers.
% Phylogenetic analysis of putative single-copy orthologs conserved in the three primary aculeate lineages (ants, bees, and vespid wasps) was unable to settle the unresolved evolutionary lineage of this clade.
% Finally, we discovered that \textit{Polistes} lacks the \textit{Dnmt3} DNA methyltransferase enzyme and has essentially zero DNA methylation genome-wide, calling into question the conventional wisdom that DNA methylation is critical to the evolution of social behavior.
% This finding was independently confirmed in a study of another wasp from the same genus, \textit{Polistes canadensis}, published just weeks before our study was submitted for peer review.
%
% The ParsEval tool addresses the fundamental need for data comparison: a need we encountered while annotating the paper wasp genome, and a need which is common to all genome projects.
% The similarity statistics reported by ParsEval help pinpoint the ways in which different sets of annotations differ, facilitating the annotation process and particularly the selection parameters.
% The locus-level detail and graphical reports produced by ParsEval are particularly beneficial in interpreting differences in annotation sets.
%
% Investigating genome composition, comparative genomics questions, gene expression, and DNA methylation in the \textit{P. dominula} genome project required precise and careful handling of genome annotation data.
% The iLocus framework was developed initially to address these issues.
% iLoci provide a straightforward mechanism for determining the proportion of the genome occupied by various types of elements, facilitating the comparison of gene content and overall genome composition across species.
% iLoci also defined the coordinate system for our analysis of gene expression, specifying an unambiguous handling of overlapping gene models and unit of quality control.
% We later generalized these concepts and applied them to a wide variety of model organisms, describing the range of genome composition and genome organization exhibited across eukaryotic diversity.
% iLoci furnish a well-defined notion of genome ``compactness" that is particularly consistent between different genome sequences from the same species and between different species within the same clade.
% Finally, iLoci provide stability between different assembly and annotation versions, facilitating reproducibility as genome assemblies and annotations are refined over time.
%
% One direct extension of the work described in this dissertation is motivated by the reproducibility challenges introduced as annotated genome assemblies are improved over time.
% Refinement of our \textit{Polistes dominula} assembly and annotation hindered progress on our research at times, and this issue is certainly not unique to our work.
% For example, the honeybee \textit{Apis mellifera} (the model social insect) had three official annotation versions in concurrent use during our work on the \textit{Polistes} genome project, with some studies even making their own unpublished refinements to an annotation \cite{Dnmt3KD,TrueSight}.
% What seemed to be lacking was a precise way to refer to a particular gene or genomic region, as annotated at a particular time, and to make statements about its expression, or conservation, or methylation status, or any number of additional characteristics.
% The idea of tracking annotations over time is not new \cite{AED}, and some well-supported communities provide tools for mapping annotations from an older assembly version to updated assembly \cite{liftOver,PAGIT}.
% iLoci provide an alternative solution to these issues and, along with the additional benefits previously described, furnish a complete framework for organization, quality control, and reproducibility for provisional genome projects.
% I have prototyped a software tool for maintaining an annotation version history, leveraging existing version control tools to track changes to individual iLoci over time.
% Supplementing this tool with quality control features for filtering iLoci based on their length, nucleotide composition, annotation quality, or other characteristics would provide a useful tool for selecting finely tuned data sets for analysis of genome features.
