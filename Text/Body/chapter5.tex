\chapter{CONCLUSIONS}

The focus of this dissertation has been the development of tools and methodologies for reproducible analysis of genome annotations, including and especially those associated with NGS-based genome projects.
The motivation for much of this work was provided by challenges encountered during the assembly, annotation, and analysis of the genome of the paper wasp \textit{Polistes dominula}.
In addressing these challenges, we sought to create tools that could be applied more generally to address common questions about genome composition, and organization.

In the \textit{P. dominula} genome project, we developed a variety of genomic data resources \textit{de novo} for investigating genome composition, gene expression, and characteristics of DNA methylation.
Our study revealed that the paper wasp genome is similar to other social insects across a variety of measures: genome size, number of annotated genes, and exon and intron characteristics, but that its nucleotide composition is the most AT-rich (aggregated over large genomic sequences) of any hymenopteran studied.
We identified 367 loci in the \textit{P. dominula} genome that are differentially expressed between queen wasps and worker wasps, and though we catalogued thousands of alternative splicing events we found no evidence of differential splicing between adult queens and workers.
Phylogenetic analysis of putative single-copy orthologs conserved in the three primary aculeate lineages (ants, bees, and vespid wasps) was unable to settle the unresolved evolutionary lineage of this clade.
Finally, we discovered that \textit{Polistes} lacks the \textit{Dnmt3} DNA methyltransferase enzyme and has essentially zero DNA methylation genome-wide, calling into question the conventional wisdom that DNA methylation is critical to the evolution of social behavior.
This finding was independently confirmed in a study of another wasp from the same genus, \textit{Polistes canadensis}, published just weeks before our study was submitted for peer review.

asdf.
